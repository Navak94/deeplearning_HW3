% Options for packages loaded elsewhere
\PassOptionsToPackage{unicode}{hyperref}
\PassOptionsToPackage{hyphens}{url}
\PassOptionsToPackage{dvipsnames,svgnames,x11names}{xcolor}
%
\documentclass[
  letterpaper,
  DIV=11,
  numbers=noendperiod]{scrartcl}

\usepackage{amsmath,amssymb}
\usepackage{iftex}
\ifPDFTeX
  \usepackage[T1]{fontenc}
  \usepackage[utf8]{inputenc}
  \usepackage{textcomp} % provide euro and other symbols
\else % if luatex or xetex
  \usepackage{unicode-math}
  \defaultfontfeatures{Scale=MatchLowercase}
  \defaultfontfeatures[\rmfamily]{Ligatures=TeX,Scale=1}
\fi
\usepackage{lmodern}
\ifPDFTeX\else  
    % xetex/luatex font selection
\fi
% Use upquote if available, for straight quotes in verbatim environments
\IfFileExists{upquote.sty}{\usepackage{upquote}}{}
\IfFileExists{microtype.sty}{% use microtype if available
  \usepackage[]{microtype}
  \UseMicrotypeSet[protrusion]{basicmath} % disable protrusion for tt fonts
}{}
\makeatletter
\@ifundefined{KOMAClassName}{% if non-KOMA class
  \IfFileExists{parskip.sty}{%
    \usepackage{parskip}
  }{% else
    \setlength{\parindent}{0pt}
    \setlength{\parskip}{6pt plus 2pt minus 1pt}}
}{% if KOMA class
  \KOMAoptions{parskip=half}}
\makeatother
\usepackage{xcolor}
\setlength{\emergencystretch}{3em} % prevent overfull lines
\setcounter{secnumdepth}{-\maxdimen} % remove section numbering
% Make \paragraph and \subparagraph free-standing
\makeatletter
\ifx\paragraph\undefined\else
  \let\oldparagraph\paragraph
  \renewcommand{\paragraph}{
    \@ifstar
      \xxxParagraphStar
      \xxxParagraphNoStar
  }
  \newcommand{\xxxParagraphStar}[1]{\oldparagraph*{#1}\mbox{}}
  \newcommand{\xxxParagraphNoStar}[1]{\oldparagraph{#1}\mbox{}}
\fi
\ifx\subparagraph\undefined\else
  \let\oldsubparagraph\subparagraph
  \renewcommand{\subparagraph}{
    \@ifstar
      \xxxSubParagraphStar
      \xxxSubParagraphNoStar
  }
  \newcommand{\xxxSubParagraphStar}[1]{\oldsubparagraph*{#1}\mbox{}}
  \newcommand{\xxxSubParagraphNoStar}[1]{\oldsubparagraph{#1}\mbox{}}
\fi
\makeatother


\providecommand{\tightlist}{%
  \setlength{\itemsep}{0pt}\setlength{\parskip}{0pt}}\usepackage{longtable,booktabs,array}
\usepackage{calc} % for calculating minipage widths
% Correct order of tables after \paragraph or \subparagraph
\usepackage{etoolbox}
\makeatletter
\patchcmd\longtable{\par}{\if@noskipsec\mbox{}\fi\par}{}{}
\makeatother
% Allow footnotes in longtable head/foot
\IfFileExists{footnotehyper.sty}{\usepackage{footnotehyper}}{\usepackage{footnote}}
\makesavenoteenv{longtable}
\usepackage{graphicx}
\makeatletter
\def\maxwidth{\ifdim\Gin@nat@width>\linewidth\linewidth\else\Gin@nat@width\fi}
\def\maxheight{\ifdim\Gin@nat@height>\textheight\textheight\else\Gin@nat@height\fi}
\makeatother
% Scale images if necessary, so that they will not overflow the page
% margins by default, and it is still possible to overwrite the defaults
% using explicit options in \includegraphics[width, height, ...]{}
\setkeys{Gin}{width=\maxwidth,height=\maxheight,keepaspectratio}
% Set default figure placement to htbp
\makeatletter
\def\fps@figure{htbp}
\makeatother

\usepackage{fvextra}
\DefineVerbatimEnvironment{Highlighting}{Verbatim}{breaklines,commandchars=\\\{\}}
\KOMAoption{captions}{tableheading}
\usepackage{multicol}
\newcommand{\btwocol}{\begin{multicols}{2}}
\newcommand{\etwocol}{\end{multicols}}
\makeatletter
\@ifpackageloaded{caption}{}{\usepackage{caption}}
\AtBeginDocument{%
\ifdefined\contentsname
  \renewcommand*\contentsname{Table of contents}
\else
  \newcommand\contentsname{Table of contents}
\fi
\ifdefined\listfigurename
  \renewcommand*\listfigurename{List of Figures}
\else
  \newcommand\listfigurename{List of Figures}
\fi
\ifdefined\listtablename
  \renewcommand*\listtablename{List of Tables}
\else
  \newcommand\listtablename{List of Tables}
\fi
\ifdefined\figurename
  \renewcommand*\figurename{Figure}
\else
  \newcommand\figurename{Figure}
\fi
\ifdefined\tablename
  \renewcommand*\tablename{Table}
\else
  \newcommand\tablename{Table}
\fi
}
\@ifpackageloaded{float}{}{\usepackage{float}}
\floatstyle{ruled}
\@ifundefined{c@chapter}{\newfloat{codelisting}{h}{lop}}{\newfloat{codelisting}{h}{lop}[chapter]}
\floatname{codelisting}{Listing}
\newcommand*\listoflistings{\listof{codelisting}{List of Listings}}
\makeatother
\makeatletter
\makeatother
\makeatletter
\@ifpackageloaded{caption}{}{\usepackage{caption}}
\@ifpackageloaded{subcaption}{}{\usepackage{subcaption}}
\makeatother

\ifLuaTeX
  \usepackage{selnolig}  % disable illegal ligatures
\fi
\usepackage{bookmark}

\IfFileExists{xurl.sty}{\usepackage{xurl}}{} % add URL line breaks if available
\urlstyle{same} % disable monospaced font for URLs
\hypersetup{
  pdftitle={CS/DS 541: Deep Learning},
  pdfauthor={Tom Arnold \& Nate Hindman},
  colorlinks=true,
  linkcolor={blue},
  filecolor={Maroon},
  citecolor={Blue},
  urlcolor={Blue},
  pdfcreator={LaTeX via pandoc}}


\title{CS/DS 541: Deep Learning}
\usepackage{etoolbox}
\makeatletter
\providecommand{\subtitle}[1]{% add subtitle to \maketitle
  \apptocmd{\@title}{\par {\large #1 \par}}{}{}
}
\makeatother
\subtitle{Homework 3}
\author{Tom Arnold \& Nate Hindman}
\date{}

\begin{document}
\maketitle

\RecustomVerbatimEnvironment{verbatim}{Verbatim}{
  showspaces = false,
  showtabs = false,
  breaksymbolleft={},
  breaklines
  % Note: setting commandchars=\\\{\} here will cause an error
}


\emph{Due: 5:59pm ET Monday September 29}

\emph{This problem can be done in teams of up 2 students.}

\section{Problem 1: Deriving He's initialization {[}10 points + 5 bonus
pts{]}}\label{problem-1-deriving-hes-initialization-10-points-5-bonus-pts}

In Lecture 7, we learned about a widely used technique for setting the
initial values of the weight parameters in a neural network: the
\textbf{He initialization}. Specifically, it samples each weight value
from \(\mathcal{N}(0,2/n_{l})\)---i.e., a 0-mean Gaussian distribution
with variance \(2/n_{l}\) where \(n_{l}\) is the number of columns of
the weight matrix \(W^{[l]}\) layer \(l\) (or, equivalently the
dimension of the inputs fed to layer \(l\)).

Here you are asked to derive some of the steps we skipped in class. We
will start from:
\[Var[z^{[l]}]=Var[w^{[l]}x^{[l]}]=Var[\sum_{j=1}^{n_{l}}w_{j}^{[l]}x_{j}^{[l]}]\]
\[Var[\sum_{j=1}^{n_{l}}w_{j}^{[l]}x_{j}^{[l]}]=n_{l}Var[w_{l}x_{l}],\]

\begin{enumerate}
\def\labelenumi{\arabic{enumi}.}
\tightlist
\item
  (2 points) Explain what are the assumptions that allow us to write:
  \[Var[\sum_{j=1}^{n_{l}}w_{j}^{[l]}x_{j}^{[l]}]=n_{l}Var[w_{l}x_{l}],\]
  where \(w_{l}\) represents any of the \(w_{j}^{[l]}\) weights and
  \(x_{l}\) represents any of the \(x_{j}^{[l]}\) inputs.
\end{enumerate}

\begin{center}\rule{0.5\linewidth}{0.5pt}\end{center}

\subsection{Answer}\label{answer}

We are given that He initialization samples weights from
\(\mathcal{N}(0, 2/n_l)\) which means that each weight has expectation
and variance:

\[E[w_j^{[l]}] = 0 \text{ and } \text{Var}[w_j^{[l]}] = 2/n_l \quad \text{(definition of } \mathcal{N}(\mu, \sigma^2))\]

\[\text{Var}\left[\sum_{j=1}^{n_l} w_j^{[l]} x_j^{[l]}\right] = E\left[\left(\sum_{j=1}^{n_l} w_j^{[l]} x_j^{[l]}\right)^2\right] - \left(E\left[\sum_{j=1}^{n_l} w_j^{[l]} x_j^{[l]}\right]\right)^2\]
\[\text{(Definition of Variance: } \text{Var}[Y] = E[Y^2] - (E[Y])^2)\]

\subsubsection{\texorpdfstring{Assumption 1: \(w_j^{[l]}\) independent
of all
\(x_k^{[l]}\)}{Assumption 1: w\_j\^{}\{{[}l{]}\} independent of all x\_k\^{}\{{[}l{]}\}}}\label{assumption-1-w_jl-independent-of-all-x_kl}

To move forward, we need to assume that the weights \(w_j^{[l]}\) are
independent of the inputs \(x_j^{[l]}\). This assumption allows us to
factor terms of the form
\(E[w_j^{[l]} x_j^{[l]}] = E[w_j^{[l]}] \cdot E[x_j^{[l]}].\) Because we
know from initialization that \(E[w_j^{[l]}] = 0\), the full product is
0 and drops out of the equation.

We are justified in this assumption because the random initialization of
weights at layer \(l\) is done before the network ever sees data, so the
particular input values flowing through the network cannot influence the
weights. Without this assumption, we would have to deal with the full
joint distribution of \((w_j^{[l]}, x_j^{[l]})\).

\[E\left[\sum_{j=1}^{n_l} w_j^{[l]} x_j^{[l]}\right] = \sum_{j=1}^{n_l} E[w_j^{[l]} x_j^{[l]}] \quad \text{(Linearity of Expectation)}\]

\[= \sum_{j=1}^{n_l} E[w_j^{[l]}] E[x_j^{[l]}] \quad \text{(Independence Theorem: } X \perp Y \Rightarrow E[XY] = E[X]E[Y])\]

\[= \sum_{j=1}^{n_l} 0 \cdot E[x_j^{[l]}] = 0\]

\[\text{Var}\left[\sum_{j=1}^{n_l} w_j^{[l]} x_j^{[l]}\right] = E\left[\left(\sum_{j=1}^{n_l} w_j^{[l]} x_j^{[l]}\right)^2\right] - 0^2\]

\[= E\left[\sum_{j=1}^{n_l}\sum_{k=1}^{n_l} w_j^{[l]} x_j^{[l]} w_k^{[l]} x_k^{[l]}\right] \quad \text{(Expanding } (a_1 + \cdots + a_n)^2 = \sum_j \sum_k a_j a_k)\]

\[= \sum_{j=1}^{n_l}\sum_{k=1}^{n_l} E[w_j^{[l]} x_j^{[l]} w_k^{[l]} x_k^{[l]}] \quad \text{(Linearity of Expectation)}\]

\subsubsection{\texorpdfstring{Assumption 2: \(w_j^{[l]}\) independent
of \(w_k^{[l]}\) for
\(j \neq k\)}{Assumption 2: w\_j\^{}\{{[}l{]}\} independent of w\_k\^{}\{{[}l{]}\} for j \textbackslash neq k}}\label{assumption-2-w_jl-independent-of-w_kl-for-j-neq-k}

When we expand the squared sum \((\sum_j w_j x_j)^2\), cross-terms of
the form \(E[w_j x_j w_k x_k]\) appear for \(j \neq k\).

\[
\text{Var}\!\left[\sum_{j=1}^{n_l} w_j x_j \right]
= 
E\!\left[
\begin{bmatrix}
w_1 x_1 & \cdots & w_{n_l} x_{n_l}
\end{bmatrix}
\begin{bmatrix}
w_1 x_1 \\ \vdots \\ w_{n_l} x_{n_l}
\end{bmatrix}
\right]
=
\begin{bmatrix}
\text{Var}[w_1 x_1] & \cdots & E[w_j x_j w_k x_k] & \cdots \\
\vdots & \ddots & \vdots & \\
E[w_k x_k w_j x_j] & \cdots & \text{Var}[w_{n_l} x_{n_l}]
\end{bmatrix}
\]

We can't move forward if these terms exist. Therefore we assume that the
weights themselves are independent, meaning knowledge of \(w_1^{[l]}\)
tells us nothing about \(w_2^{[l]}\), and so on. This independence
causes these cross-terms to be 0, leaving only the diagonal
contributions.

Thus, for \(j \neq k\):
\[E[w_j^{[l]} x_j^{[l]} w_k^{[l]} x_k^{[l]}] = E[w_j^{[l]}]E[x_j^{[l]}]E[w_k^{[l]}]E[x_k^{[l]}] = 0 \cdot E[x_j^{[l]}] \cdot 0 \cdot E[x_k^{[l]}] = 0\]
\[\text{(Independence of all four random variables)}\]

\[= \sum_{j=1}^{n_l} E[(w_j^{[l]})^2 (x_j^{[l]})^2] \quad \text{(only diagonal terms } j = k \text{ survive)}\]

\[= \sum_{j=1}^{n_l} E[(w_j^{[l]} x_j^{[l]})^2]\]

\[= \sum_{j=1}^{n_l} \left(\text{Var}[w_j^{[l]} x_j^{[l]}] + (E[w_j^{[l]} x_j^{[l]}])^2\right) \quad \text{(Variance-Mean Relation: } E[Y^2] = \text{Var}[Y] + (E[Y])^2)\]

\[= \sum_{j=1}^{n_l} \text{Var}[w_j^{[l]} x_j^{[l]}] \quad \text{(since } E[w_j^{[l]} x_j^{[l]}] = E[w_j^{[l]}]E[x_j^{[l]}] = 0)\]

\subsubsection{\texorpdfstring{Assumption 3: All \(w_j^{[l]}\)
identically distributed, all \(x_j^{[l]}\) identically
distributed}{Assumption 3: All w\_j\^{}\{{[}l{]}\} identically distributed, all x\_j\^{}\{{[}l{]}\} identically distributed}}\label{assumption-3-all-w_jl-identically-distributed-all-x_jl-identically-distributed}

To simplify further we must assume that all these weight--input pairs
are identically distributed. This assumption is justified because under
He initialization all weights are drawn from the same distribution, and
the architecture of the network ensures that the inputs \(x_j^{[l]}\) at
a given layer have the same statistical properties. Without assuming
identical distributions, each term in the summation could be different,
forcing us to track them one by one rather than collapsing them as we do
below.

\[= n_l \cdot \text{Var}[w^{[l]} x^{[l]}] \quad \text{(Identical Distribution: all terms equal)}\]

\begin{center}\rule{0.5\linewidth}{0.5pt}\end{center}

\begin{enumerate}
\def\labelenumi{\arabic{enumi}.}
\setcounter{enumi}{1}
\tightlist
\item
  (4 points) Show that
  \[Var[w_{l}x_{l}]=Var[w_{l}]\mathbb{E}[(x_{l})^{2}]\]using the
  following equality regarding the variance of the product of two
  mutually independent random variables A and
  B:\[Var[AB]=Var[A]Var[B]+Var[A]\mathbb{E}[B]^{2}+Var[B]\mathbb{E}[A]^{2} \quad (0.0.1)\]You
  will need to: (i) review and use the assumptions made about the
  distribution of \(w_{l}\) and (ii) use the relationship between
  variance and second moment for a random variable
  B:\[Var[B]=\mathbb{E}[B^{2}]-\mathbb{E}[B]^{2}. \quad (0.0.2)\]
\end{enumerate}

\begin{center}\rule{0.5\linewidth}{0.5pt}\end{center}

\subsection{Answer}\label{answer-1}

\[\text{Var}[w^{[l]} x^{[l]}] = E[(w^{[l]} x^{[l]})^2] - (E[w^{[l]} x^{[l]}])^2 \quad \text{(Definition of Variance)}\]

\[E[w^{[l]} x^{[l]}] = E[w^{[l]}] \cdot E[x^{[l]}] \quad \text{(Independence Theorem)}\]

\[= 0 \cdot E[x^{[l]}] = 0 \quad \text{(using } E[w^{[l]}] = 0 \text{ from He initialization)}\]

\[\text{Var}[w^{[l]} x^{[l]}] = E[(w^{[l]} x^{[l]})^2] - 0^2 = E[(w^{[l]})^2 (x^{[l]})^2]\]

\[= E[(w^{[l]})^2] \cdot E[(x^{[l]})^2] \quad \text{(Independence Theorem for products)}\]

\[E[(w^{[l]})^2] = \text{Var}[w^{[l]}] + (E[w^{[l]}])^2 \quad \text{(Variance-Mean Relation)}\]

\[= \text{Var}[w^{[l]}] + 0^2 = \text{Var}[w^{[l]}]\]

\[\therefore \text{Var}[w^{[l]} x^{[l]}] = \text{Var}[w^{[l]}] \cdot E[(x^{[l]})^2]\]

\begin{center}\rule{0.5\linewidth}{0.5pt}\end{center}

\begin{enumerate}
\def\labelenumi{\arabic{enumi}.}
\setcounter{enumi}{2}
\tightlist
\item
  Last, you need to relate the variance of the input \(x_{l}\) with the
  variance of the pre-activation value \(z_{l-1}\) when the activation
  function is \textbf{ReLU}, i.e.~when
  \(x_{l}=ReLU(z_{l-1})=ReLU(w_{l-1}x_{l-1}+b_{l-1})\). Specifically,
  you need to prove the relationship:
  \[Var[z_{l-1}]=2\times\mathbb{E}[x_{l}^{2}].\] To do so, you will need
  to use these assumptions:
\end{enumerate}

\begin{itemize}
\tightlist
\item
  \(w_{l-1}\) has a \textbf{symmetric distribution around 0}, i.e., the
  density \(f_{w_{l-1}}(w)=f_{w_{l-1}}(-w).\)
\item
  \(b_{l-1}=0\)
\end{itemize}

As a guide, try to answer these questions in order:

\begin{enumerate}
\def\labelenumi{(\alph{enumi})}
\item
  (2 points) What is the expected value of \({z_{l-1}}\)? Does it depend
  on the value of \({x_{l-1}}?\)
\item
  (1 point) Think of \(x_{l-1}\) as a constant. What is the density
  \(f_{z_{l-1}}(z)!\)
\item
  (1 point) Is the distribution of \(z_{l-1}\) symmetric around 0?
\item
  (BONUS: 3 points) The variance of a continuous random variable X is
  given by \(\int_{-\infty}^{\infty}(x-\mathbb{E}[X])^{2}f_{X}(x)dx.\)
  Write an expression for \(Var[z_{l-1}]\). Remember what you found in
  item (a) about \(\mathbb{E}[z_{l-1}]\) and try to use the symmetry of
  the distribution to write \(Var[z_{l-1}]\) only in terms of the
  positive values of \(z_{l-1}\).
\item
  (BONUS: 2 points) Last, starting from the definition of the second
  moment
  \(\mathbb{E}[x_{l}^{2}]=\int_{-\infty}^{\infty}x_{l}^{2}f_{x_{l}}(x)d_{x}\)
  , find ways to replace \(x_{l}\) by \(z_{l-1}\) conditioning on
  whether \(z_{l-1}\) is negative or positive. You should be able to
  find the expression you derived in item (d) for \(Var[z_{l-1}]\).
\end{enumerate}

\begin{center}\rule{0.5\linewidth}{0.5pt}\end{center}

\subsection{Answer}\label{answer-2}

\subsubsection{\texorpdfstring{(a) Expected value of
\(z_{l-1}\):}{(a) Expected value of z\_\{l-1\}:}}\label{a-expected-value-of-z_l-1}

Given: \(b_{l-1} = 0\)

\[z_{l-1} = \sum_{k=1}^{n_{l-1}} w_k^{[l-1]} x_k^{[l-1]}\]

\[E[z_{l-1}] = E\left[\sum_{k=1}^{n_{l-1}} w_k^{[l-1]} x_k^{[l-1]}\right] = \sum_{k=1}^{n_{l-1}} E[w_k^{[l-1]} x_k^{[l-1]}] \quad \text{(Linearity of Expectation)}\]

\[= \sum_{k=1}^{n_{l-1}} E[w_k^{[l-1]}] \cdot E[x_k^{[l-1]}] \quad \text{(Independence Theorem)}\]

\[= \sum_{k=1}^{n_{l-1}} 0 \cdot E[x_k^{[l-1]}] = 0 \quad \text{(He initialization: } E[w_k^{[l-1]}] = 0)\]

\subsubsection{\texorpdfstring{(b) Density of \(z_{l-1}\) given
\(x_{l-1}\):}{(b) Density of z\_\{l-1\} given x\_\{l-1\}:}}\label{b-density-of-z_l-1-given-x_l-1}

\[z_{l-1} = w_{l-1} \cdot x_{l-1} \quad \text{(treating } x_{l-1} \text{ as constant scalar)}\]

\[f_{z_{l-1}|x_{l-1}}(z) = \frac{1}{|x_{l-1}|} f_{w_{l-1}}\left(\frac{z}{x_{l-1}}\right)\]
\[\text{(Change of Variables Formula: if } Y = aX, \text{ then } f_Y(y) = \frac{1}{|a|}f_X(y/a))\]

\subsubsection{(c) Symmetry:}\label{c-symmetry}

Given: \(w_{l-1} \sim \mathcal{N}(0, \sigma^2)\)
\[\Rightarrow f_{w_{l-1}}(w) = \frac{1}{\sqrt{2\pi\sigma^2}} e^{-w^2/(2\sigma^2)} = f_{w_{l-1}}(-w)\]
\[\text{(Gaussian PDF is symmetric about its mean)}\]

\[f_{z_{l-1}}(z) = f_{z_{l-1}}(-z) \quad \text{(Linear transformation preserves symmetry)}\]

\subsubsection{(d) Variance calculation:}\label{d-variance-calculation}

\[\text{Var}[z_{l-1}] = E[z_{l-1}^2] - (E[z_{l-1}])^2 = E[z_{l-1}^2] - 0 \quad \text{(Definition of Variance)}\]

\[E[z_{l-1}^2] = \int_{-\infty}^{\infty} z^2 f_{z_{l-1}}(z) dz \quad \text{(Definition of Expectation for continuous RV)}\]

\[= \int_{-\infty}^{0} z^2 f_{z_{l-1}}(z) dz + \int_{0}^{\infty} z^2 f_{z_{l-1}}(z) dz \quad \text{(Additivity of Integrals)}\]

\[= \int_{0}^{\infty} u^2 f_{z_{l-1}}(-u) du + \int_{0}^{\infty} z^2 f_{z_{l-1}}(z) dz\]
\[\text{(U-Substitution Rule for Integrals: Let $u = -z$ in first integral, then $du = -dz$)}\]

\[= \int_{0}^{\infty} u^2 f_{z_{l-1}}(u) du + \int_{0}^{\infty} z^2 f_{z_{l-1}}(z) dz\]
\[\text{(Using symmetry: } f_{z_{l-1}}(-u) = f_{z_{l-1}}(u))\]

\[= 2 \int_{0}^{\infty} z^2 f_{z_{l-1}}(z) dz \quad \text{(Both integrals are identical)}\]

\subsubsection{\texorpdfstring{(e) Connecting to
\(E[x_l^2]\):}{(e) Connecting to E{[}x\_l\^{}2{]}:}}\label{e-connecting-to-ex_l2}

\[x_l = \text{ReLU}(z_{l-1}) = \begin{cases} z_{l-1} & \text{if } z_{l-1} > 0 \\ 0 & \text{if } z_{l-1} \leq 0 \end{cases} \quad \text{(Definition of ReLU)}\]

\[E[x_l^2] = \int_{-\infty}^{\infty} [\text{ReLU}(z)]^2 f_{z_{l-1}}(z) dz \quad \text{(Definition of Expectation)}\]

\[= \int_{-\infty}^{0} 0^2 \cdot f_{z_{l-1}}(z) dz + \int_{0}^{\infty} z^2 f_{z_{l-1}}(z) dz\]
\[\text{(ReLU}(z) = 0 \text{ for } z \leq 0, \text{ ReLU}(z) = z \text{ for } z > 0)\]

\[= 0 + \int_{0}^{\infty} z^2 f_{z_{l-1}}(z) dz = \int_{0}^{\infty} z^2 f_{z_{l-1}}(z) dz\]

From part (d):
\(\text{Var}[z_{l-1}] = 2 \int_{0}^{\infty} z^2 f_{z_{l-1}}(z) dz\)

\[\therefore \text{Var}[z_{l-1}] = 2 \cdot E[x_l^2]\]

\begin{center}\rule{0.5\linewidth}{0.5pt}\end{center}

\section{Problem 2: Training NNs with pytorch {[}15 points + 2 bonus
pts{]}}\label{problem-2-training-nns-with-pytorch-15-points-2-bonus-pts}

In this problem you will use pytorch to train a multi-layer neural
network to classify images of fashion items (10 different classes) from
the \textbf{Fashion MNIST} dataset. Similarly to Homework 2, the input
to the network will be a \(28\times28\) pixel image; the output will be
a real number.

Specifically, the starting network you will create should implement a
function \(f:\mathbb{R}^{784}\rightarrow\mathbb{R}^{10},\) where:
\[\begin{aligned}z^{(1)}&=W^{(1)}x+b^{(1)}\\ h^{(1)}&=ReLU(z^{(1)})\\ z^{(2)}&=W^{(2)}h^{(1)}+b^{(2)}\\ &:\end{aligned}\]
\[\begin{aligned}z^{(l)}&=W^{(l)}h^{(l-1)}+b^{(l)}\\ \hat{y}&=softmax(z^{(l)})\end{aligned}\]
The network specified above is shown in the figure below: X
\(\rightarrow\) \(W^{(1)}\) \(\rightarrow\) \(b^{(1)}\) \(\rightarrow\)
\(z^{(1)}\) \(\rightarrow\) \(h^{(1)}\) \(\rightarrow\) \(W^{(2)}\)
\(\rightarrow\) \(\mathfrak{D}^{(2)}\) \(\rightarrow\) \(z^{(2)}\)
\(h^{(2)}\) \(\rightarrow\) \ldots{} \(\rightarrow\) \(z^{(l)}\)
\(\rightarrow\) \(W^{(l)}\)/ \(b^{(l)}\) \(\rightarrow\) \(\hat{y}\)

As usual, the (unregularized) cross-entropy cost function should be:
\[f_{CE}(W^{(1)},b^{(1)},...,W^{(l)},b^{(l)})=-\frac{1}{n}\sum_{i=1}^{n}\sum_{k=1}^{10}y_{k}^{(i)}log\hat{y}_{k}^{(i)}\]
where \(n\) is the number of examples.

The Fashion MNIST dataset can be obtained from the following web links:
\url{https://s3.amazonaws.com/jrwprojects/fashion_mnist_train_images.npy}
\url{https://s3.amazonaws.com/jrwprojects/fashion_mnist_train_labels.npy}
\url{https://s3.amazonaws.com/jrwprojects/fashion_mnist_test_images.npy}
\url{https://s3.amazonaws.com/jrwprojects/fashion_mnist_test_labels.npy}

\begin{enumerate}
\def\labelenumi{\arabic{enumi}.}
\item
  (6 points) Implement the same network using PyTorch or Tensorflow.
\item
  (5 points) Implement a method called \textbf{hyperparam\_tuning}.
  Optimize the hyperparameters by training on the training set and
  selecting the parameter settings that optimize performance on the
  validation set. You should systematically (i.e., in code) try at least
  \textbf{10} (in total, not for each hyperparameter) different
  hyperparameter settings; \textbf{Hyperparameter tuning:} In this
  problem, there are several different hyperparameters that will impact
  the network's performance:

  \begin{itemize}
  \tightlist
  \item
    (Required) \textbf{Number of hidden layers} (suggestions:
    \(\{3,4,5\}\))
  \item
    \textbf{Number of units in each hidden layer} (suggestions:
    \(\{30,40,50\}\))
  \item
    \textbf{Learning rate} (suggestions:
    \(\{0.001, 0.005, 0.01, 0.05, 0.1, 0.5\}\))
  \item
    \textbf{Minibatch size} (suggestions: \(\{16,32, 64, 128, 256\}\))
  \item
    \textbf{Number of epochs}
  \item
    \(L_{2}\) \textbf{Regularization strength} applied to the weight
    matrices (but not bias terms) In order not to ``cheat'' and thus
    overestimate the performance of the network it is crucial to
    optimize the hyperparameters \textbf{only on the validation set}; do
    \textbf{not} use the test set. (The training set would be ok, but
    typically leads to worse performance.) Hence, just like in Homework
    2, you should fork off part of the training set into a validation
    portion.
  \end{itemize}
\item
  (4 points) After you have optimized your hyperparameters, then
  re-train your network again and show a screenshot displaying the
  training loss evolving over the last \textbf{20 iterations} of SGD.
  (You can pick the last 20 mini-batches or last 20 epochs; it's up to
  you). In addition, make sure your screenshot also displays the
  \textbf{test accuracy} of the final trained classifier. The accuracy
  (percentage correctly classified test images) should be \textbf{at
  least 87\%}.
\item
  (2 bonus points) Now you have to experiment with a variant of the
  previous network that includes either \textbf{BatchNorm, Dropout
  layers} or \textbf{different activation functions} (e.g., LeakyReLU,
  ELU or PReLU). Find an architecture that outperforms the one you found
  in item 3 and report the test accuracy.
\end{enumerate}

\begin{center}\rule{0.5\linewidth}{0.5pt}\end{center}

\section{Problem 3: Visualizing the loss landscape and optimization
trajectories (15
points)}\label{problem-3-visualizing-the-loss-landscape-and-optimization-trajectories-15-points}

Visualize the SGD trajectory of your network when trained on Fashion
MNIST:

\begin{enumerate}
\def\labelenumi{\arabic{enumi}.}
\item
  Plot in 3-D the cross-entropy loss \(f_{CE}\) as a function of the
  neural network parameters (weights and bias terms). Rather than
  showing the loss as a function of any particular parameters (e.g., the
  third component of \(b^{(2)}\)), use the \textbf{x and y-axes} to span
  the two directions along which your parameters vary the most during
  SGD training i.e., you will need to use \textbf{principal component
  analysis (PCA)}. (In this assignment, you are free to use
  \texttt{sklearn.decomposition.PCA}.) The \textbf{z-axis} should
  represent the (unregularized) \(f_{CE}\) on training data. For each
  point \((x,y)\) on a grid, compute \(f_{CE}\) and then interpolate
  between the points. (The interpolation and rendering are handled for
  you completely by the \texttt{plot\_surface} function; see the starter
  code).
\item
  Superimpose a scatter plot of the different points (in the neural
  network's parameter space) that were reached during SGD (just sample a
  few times per epoch). To accelerate the rendering, train the network
  and plot the surface using just a small subset of the training data
  (e.g., 2500 examples) to estimate \(f_{CE}\). See the starter code, in
  particular the \texttt{plotPath} function, for an example of 3-D
  plotting. Submit your graph in the PDF file and the code in the Python
  file.

  Here is what I get when I superimpose two scatter plots corresponding
  to two different initializations and SGD runs (note that you only need
  to include a scatter plot for one run); as you can see, the two SGD
  trajectories descended into distinct local minima (though with similar
  cost):
\end{enumerate}

\begin{center}\rule{0.5\linewidth}{0.5pt}\end{center}

\section{Problem 4: Autoencoder (20
points)}\label{problem-4-autoencoder-20-points}

This assignment explores one powerful concept in deep learning:
\textbf{unsupervised representation learning with autoencoders}. You
will build a neural network that learns to compress and then reconstruct
images, forcing it to learn a meaningful representation of the data in
the process.

\subsection{Task 1: Data Preparation (2
point)}\label{task-1-data-preparation-2-point}

\begin{enumerate}
\def\labelenumi{\arabic{enumi}.}
\tightlist
\item
  Load the \textbf{Fashion MNIST} dataset using
  \texttt{torchvision.datasets.FashionMNIST}.
\item
  Use \texttt{torchvision.transforms.ToTensor} to convert the images to
  PyTorch tensors and normalize the pixel values to be in the range
  \([0.0, 1.0]\).
\item
  Create a \texttt{DataLoader} for both the training and test sets. For
  this unsupervised task, we only need the image data, not the labels.
\end{enumerate}

\subsection{Task 2: Autoencoder Architecture (6
points)}\label{task-2-autoencoder-architecture-6-points}

Construct a \textbf{dense autoencoder model} by creating a class that
inherits from \texttt{torch.nn.Module}. The model should consist of an
\textbf{encoder} and a \textbf{decoder} with the following symmetric
architecture:

\begin{itemize}
\item
  \textbf{An Encoder} that takes a flattened 784-dimensional input and
  maps it to a compressed representation.

  \begin{itemize}
  \tightlist
  \item
    Input layer (implicitly defined by the input shape).
  \item
    Dense layer (\texttt{nn.Linear}) with \textbf{128 units} followed by
    a \textbf{ReLU} activation (\texttt{nn.ReLU}).
  \item
    Dense layer with \textbf{64 units} followed by a \textbf{ReLU}
    activation. This layer's output is the \textbf{latent
    representation}, also known as the ``\textbf{bottleneck}''.
  \end{itemize}
\item
  \textbf{A Decoder} that takes the 64-dimensional latent representation
  and reconstructs the 784-dimensional image.

  \begin{itemize}
  \tightlist
  \item
    Dense layer with \textbf{128 units} followed by a \textbf{ReLU}
    activation.
  \item
    Dense layer with \textbf{784 units} followed by a \textbf{Sigmoid}
    activation (\texttt{nn.Sigmoid}). The sigmoid activation ensures the
    output values are in the range \([0.0, 1.0]\), matching the
    normalized input data.
  \end{itemize}
\end{itemize}

\subsection{Task 3: Model Training (6
points)}\label{task-3-model-training-6-points}

\begin{enumerate}
\def\labelenumi{\arabic{enumi}.}
\tightlist
\item
  Instantiate your model, a loss function (\texttt{nn.MSELoss} is a good
  choice), and an optimizer (\texttt{torch.optim.Adam}).
\item
  Write a training loop that iterates for \textbf{20 epochs}. In each
  epoch, iterate through the training \texttt{DataLoader}.
\item
  For each batch of images, you must:

  \begin{itemize}
  \tightlist
  \item
    Flatten the \(28\times28\) images into 784-dimensional vectors.
  \item
    Perform a forward pass to get the reconstructed images.
  \item
    Calculate the loss between the original and reconstructed images.
  \item
    Zero the gradients, perform a backward pass, and update the weights.
  \end{itemize}
\end{enumerate}

\subsection{Task 4: Visualizing Reconstructions (6
points)}\label{task-4-visualizing-reconstructions-6-points}

\begin{enumerate}
\def\labelenumi{\arabic{enumi}.}
\tightlist
\item
  Set your model to \textbf{evaluation mode}.
\item
  Get a batch of images from the test set and pass them through your
  trained autoencoder to get the reconstructions.
\item
  Create a plot showing \textbf{two different original images} from the
  test set and their corresponding reconstructions directly below them.
\item
  \textbf{Deliverable:} A single figure containing 4 images (2 original
  test images, 2 reconstructed images), clearly labeled.
\end{enumerate}

\begin{center}\rule{0.5\linewidth}{0.5pt}\end{center}

\section{Submission}\label{submission}

Submit one PDF file that includes your notes for the theoretical
problems (scanned or typed) and screenshots of your code for the
programming problems. All material in the submitted PDF must be
presented in a clear and readable format.

If you are working as part of a group, then indicate the members on
canvas:
\href{https://www.google.com/search?q=https://canvas.wpi.edu/courses/76771/groups\%23tab-14853}{https://canvas.wpi.edu/courses/76771/groups\#tab-14853}.
Once you do that, be aware that any submission from a team member will
overwrite an existing one.

\subsection{Teamwork}\label{teamwork}

You may complete this homework assignment either individually or in
teams up to 2 people.




\end{document}
